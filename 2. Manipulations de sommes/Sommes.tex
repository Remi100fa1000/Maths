\documentclass{article}

\usepackage[utf8]{inputenc}
\usepackage[T1]{fontenc}
\usepackage[francais]{babel}

\usepackage{amsmath}
\usepackage{amssymb}

\usepackage{stmaryrd}

\begin{document}
\title{Quelques manipulations de sommes}

\maketitle

\section{Objectif}

Soit $0<k \leq N $ deux entiers positifs.On cherche à réduire la somme

\begin{equation}
\label{eq:a_reduire}
\sum_{l=1}^{N-k+1} {N-l \choose k-1}l.
\end{equation}

Pour calculer cette somme, nous allons utiliser le résultat suivant :

Soit $0<k \leq N $ deux entiers positifs. On a 

\begin{equation}
\label{eq:resultat_precedent}
\sum_{l=1}^{N-k+1} {N-l \choose k-1} = {N \choose k}
\end{equation}

\section{Développements}

On cherche à transformer l'équation \eqref{eq:a_reduire} en sommes d'expressions de sorte à trouver des expressions de la forme de l'équation \eqref{eq:resultat_precedent}.

On écrit

\begin{multline}
\sum_{l=1}^{N-k+1} {N-l \choose k-1}l = k\sum_{l=1}^{N-k+1} {N-l \choose k-1} \frac{l}{k} \\
+ k\sum_{l=1}^{N-k+1} {N-l \choose k-1} \frac{(N+1)}{k} - k\sum_{l=1}^{N-k+1} {N-l \choose k-1} \frac{(N+1)}{k}. 
\end{multline}

Les deux termes introduits étant opposés, leur somme est nulle. En utilisant \eqref{eq:resultat_precedent}, on a 

\begin{equation}
(N+1)\sum_{l=1}^{N-k+1} {N-l \choose k-1} = (N+1){N \choose k}.
\end{equation}

Ensuite,

\begin{multline}
k\sum_{l=1}^{N-k+1} {N-l \choose k-1}\left( \frac{l}{k}-\frac{(N+1)}{k} \right) \\
= -k\sum_{l=1}^{N-k+1} \frac{(N+1-l)}{k}\frac{(N-l)!}{(k-1)!(N+1-l-k)!}\\
= -k \sum_{l=1}^{N+1-k}{N+1-l \choose k}.
\end{multline}

Finallement, en utilisant \eqref{eq:resultat_precedent}, on a

\begin{equation}
-k \sum_{l=1}^{N+1-k}{N+1-l \choose k} = -k {N+1 \choose k+1}.
\end{equation}

Donc,

\begin{equation}
\sum_{l=1}^{N-k+1} {N-l \choose k-1}l = (N+1){N \choose k}-k {N+1 \choose k+1}
\end{equation}


\end{document}