\documentclass{article}

\usepackage[utf8]{inputenc}
\usepackage[T1]{fontenc}
\usepackage[francais]{babel}

\usepackage{amsmath}
\usepackage{amssymb}

\usepackage{stmaryrd}

\begin{document}
\title{Un exercice de dénombrement}

\maketitle

\section{Objectif}

On montre de deux manières différentes la proposition $P$ :\newline

$P$ : Soit $0<k \leq N $ deux entiers positifs. ${N \choose k} = \sum_{l=1}^{N-k+1} {N-l \choose k-1}$.

\section{Première preuve :  par récurrence}

Soit $ k \in \mathbb{N}^* $, on cherche à montrer que la propriété $P$ est vraie pour tout $0<k \leq N$. Pour cela, on fait une récurrence sur $N$ en commençant par $N=k$ et en montrant que si la propriété est vraie pour $N$, elle est vraie pour $N+1$.

\textit{Initialisation} : l'égalité est vraie pour $N=k$, en effet,

\begin{equation*}
{k \choose k} = 1,
\end{equation*}

et,

\begin{equation*}
\sum_{l=1}^{k-k+1} {k-l \choose k-1} = {k-1 \choose k-1} = 1.
\end{equation*}

\textit{Hérédité} : on montre maintenant que, pour un $k$ donné, si la propriété est vraie au rang $N$, elle est vraie pour $N+1$. On cherche à réécrire

\begin{equation*}
\sum_{l=1}^{N+1-k+1} {N-l+1 \choose k-1},
\end{equation*}

pour cela, on fait un changement de variable $l=l-1$, on obtient

\begin{equation*}
\sum_{l=0}^{N-k+1} {N-l \choose k-1} = \sum_{l=1}^{N-k+1} {N-l \choose k-1} + {N \choose k-1}.
\end{equation*}

On utilise maintenant l'hypothèse de récurrence
\begin{equation*}
\sum_{l=1}^{N-k+1} {N-l \choose k-1} + {N \choose k-1} = {N \choose k}+ {N \choose k-1}
\end{equation*}

Finalement, on a
\begin{equation*}
{N \choose k}+ {N \choose k-1}= \frac{N!}{k!(N-k)!} + \frac{N!}{k!(N-k+1)!} = \frac{(N+1)!}{k!(N+1-k)!} = {N+1 \choose k}.
\end{equation*}

Ce qui prouve l'hérédité, clos la récurrence et prouve la proposition.

\section{Seconde preuve :  en dénombrant}

On considère un ensemble $E$ composé de $N>0$ élément. Dans cet ensemble, on cherche le nombre d'ensembles de $0<k\leq N$ éléments. Ce nombre est égal à

\begin{equation*}
{N \choose k}.
\end{equation*}

On va maintenant compter autrement le nombre d'ensembles $F$ à $k$ éléments. Pour cela on numérote de $1$ à $N$ les éléments de $E$. Pour dénombrer les ensembles de $k$ éléments, on compte ceux dont le premier élément est l'élément $l \in  \llbracket 1;N \rrbracket$.

Pour $l=1$, le premier élément de l'ensemble $F$ est en position $1$. Dans ce cas là, on a 

\begin{equation*}
{N-1 \choose k-1}
\end{equation*}

ensembles de $k$ éléments. Ensuite, p^our $l=2$, on a 

\begin{equation*}
{N-2 \choose k-1}
\end{equation*}

ensembles $F$ dont le premier élément est le deuxième de $E$. En continuant de dénombrer ainsi, on compte le nombre d'éléments $F$ qui est égal à

\begin{equation*}
\sum_{l=1}^{N-k+1} {N-l \choose k-1}
\end{equation*}

Finalement,

\begin{equation*}
{N \choose k} = \sum_{l=1}^{N-k+1} {N-l \choose k-1}.
\end{equation*}
\end{document}